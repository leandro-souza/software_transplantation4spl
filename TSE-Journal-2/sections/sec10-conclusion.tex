\section{Conclusions and Future Work} \label{sec:conclusion_future_work} 

In this work, we propose an approach, \FOUNDRY, and a tool, \autoscalpel, for product line migration by reusing existing codebases with minimal human involvement. Both approach and the tool have been validated through case studies\footnote{We will provide a replication package in case of acceptance}. We generated two product lines and two products through the transplantation of features extracted from three real-world systems into two different product bases. Additionally, we performed an experiment with SPL experts to compare our approach with manual effort, showing significant time effort improvements when using \autoscalpel. The tool accomplished the product line generation process by inserting two new variants 4.8 times faster than the mean of participants who were able to finish the experiment in the timeout.

We argued that the migration into a transplantation-based—in contrast to an annotation-based—software product line makes it usable in practice, improves maintainability due to physically separated features, and guarantees to preserve semantics.  \FOUNDRY approach can be used both to \emph{extractive} or \emph{reactive} product line migration strategy, as a systematic Clone\&own strategy to specialize existing products, avoiding the duplication of feature implementations, preserving features behavior, reduce feature redundancy, and automating changes propagation. That is, solutions for problems often cited in both SPL and clone\&own literature.

Our evaluation studies provide initial evidence to support the claim that automated product line generation, using the transplantation idea, is a feasible and, indeed, promising direction for software development with minimal human involvement. However, more studies are needed to provide more evidence for generalisability of the approach in a different domain, and that also considers its application in an industry context.
